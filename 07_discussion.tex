\section{Discussion and Limitations}

\noindent \textcolor{blue}{The ML apporaches that integrated in our system have been validated by the researchers in neuroscience~\cite{mateos2018structural,tanveer2020machine}. The features extracted from the DTI images also has been verified by research articles~\cite{acosta2016whole, wen2016white}.} Direct visualization of dis-aggregated features in the physical space can highlight underlying issues that affect the averages used for group level comparison. These patterns also have non-trivia physiological explanations that could enhance current understanding of neurodegenerative disease. As one gains insight into the important physical patterns, a rational next step would be to develop formalized descriptors that could be automatically extracted. While those important differences may be localized, precisely where they are localized, and how (or whether) they are expressed will vary between individuals. Currently, qualitative analysis through visualization is used to investigate those details, but it is a still a daunting task. Our tool will help to alleviate those difficulties, but significant challenges remain. One direction for further research is towards spatially invariant feature localisation and extraction using  neural networks (deep feature learning). As these kinds of advanced (often black box) methods are introduced, alleviating concerns about explain-ability, ease-of-use, and standardization are motivated; visualization can play a major role.

Another limitation comes from issues with data assimilation. Our current system requires the compared brain data to be obtained through identical imaging processes and also further be processed in a particular way afterwards. With more reliable techniques to assimilate data from different sources, it may be possible to greatly expand the amount of data that could be used together. With larger data, narrowing down significant differences would be easier, and more fine grained ROI/physical analysis as well as age and gender based analysis could be carried out with higher confidence. Since the state of the art in tractography is actively evolving, in the future there should be more standardized and well understood methods in use, which will appease researchers who wish to compare results between studies and maintain a comfortable understanding of the process to be confident in their judgements. 
It's noteworthy that 
current tractography can be computationally expensive. It took us about $60$ days in total to process about $190$ brain scans. Improved implementations better utilizing acceleration hardware such as GPUs will benefit the field tremendously. 


